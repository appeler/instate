\usepackage{amsfonts}

\usepackage{pstricks,fullpage, subfigure, 
  epsfig,amsthm,amsmath,latexsym,amssymb,verbatim,url,setspace,multirow,fancyhdr,
  pdfsync,rotating,xfrac,footnote,pst-plot}

\usepackage{graphicx}
\usepackage{tikz}
\usetikzlibrary{matrix}
\usepackage[bottom]{footmisc}
\usepackage[round,comma]{natbib}
%\usepackage[hidelinks]{hyperref}
\usepackage{rotating}
\usepackage[bottom]{footmisc}
\usepackage{bbm}
\usepackage[font=footnotesize,labelfont=bf]{caption}
% \usepackage{subcaption}

\usepackage{lscape}
\newcommand{\bland}{\begin{landscape}}
\newcommand{\eland}{\end{landscape}}

\setcounter{MaxMatrixCols}{10}

\bibliographystyle{ecta}
\usepackage{hyperref}

\hypersetup{
   colorlinks=true,           linkcolor=black,
   citecolor=black,            filecolor=magenta,
   urlcolor=darkgray}



\newcommand{\post}{\textsf{post}}
\newcommand{\ante}{\textsf{ante}}
\newcommand{\noi}{\noindent}
\newcommand{\sgn}{\textsf{sgn}\,}
\newcommand{\card}{\textrm{card}}
\newcommand{\supp}{\textsf{supp}\,}
\newcommand{\interior}{\textsf{int}\,}
\newcommand{\comp}{\textsf{comp}}
\newcommand{\coll}[1]{\ol{#1}}
\newcommand{\bmk}{\textsf{bmk}}
\newcommand{\mquad}{\!\!\!\!\!\!\!\!\!\!\!}
\newcommand{\mcm}{\mathcal M}
\newcommand{\mcl}[1]{\mathcal{#1}}
\newcommand{\dotP}[1]{\left< #1\right>}
\newcommand{\1}{\mathbf{1}}
\newcommand{\esp}{\mathbb{E}}
\newcommand{\ti}[1]{\tilde #1}
\newcommand{\ul}[1]{\underline{#1}}
\newcommand{\ol}[1]{\overline{#1}}
\newcommand{\ora}[1]{\mathbf{#1}}
\newcommand{\der}[2]{\frac{\partial #1}{\partial #2}}
\newcommand{\derr}[3]{\frac{\partial^2 #1}{\partial #2 \partial #3}}
\newcommand{\wh}[1]{\widehat{#1}}
\newcommand{\intd}{\textrm d}
\newcommand{\note}[1]{ \smallskip \textcolor{blue}{[note -- #1]} \smallskip }
\newcommand{\prob}{\textrm{prob}}

\newtheorem{lemma}{Lemma}
\newtheorem{theorem}{Theorem}
\newtheorem{fact}{Fact}
\newtheorem{proposition}{Proposition}
\newtheorem{definition}{Definition}
\newtheorem{assumption}{Assumption}
\newtheorem{example}{Example}
\newtheorem{corollary}{Corollary}
\newtheorem{remark}{Remark}

% \UseRawInputEncoding

\newenvironment{proofApp}[1]{\noi \textbf{Proof of #1. }}{$\quad
\blacksquare$\\}
\renewenvironment{proof}{\noi \textbf{Proof.}}{$\quad
\blacksquare$\\}
\newenvironment{proofComment}[1]{\noi \textbf{Proof (#1): }}{$\quad
\blacksquare$\\}
\newenvironment{proofIntuition}{\noi \textbf{Intuition of the proof:
}}{$\quad \blacksquare$\\}


\newenvironment{rList}{\setcounter{Lcount}{0}
\begin{list}{(\roman{Lcount}) } {\usecounter{Lcount}
\setlength{\rightmargin}{\leftmargin}}}{\end{list}}

\newcounter{Lcount}

\graphicspath{{figs/}}

\textwidth 6in \textheight 8in
\setlength{\oddsidemargin}{0.3in}
\setlength{\evensidemargin}{0.3in}
\widowpenalty10000
\clubpenalty10000
\linespread{1.075}

\usepackage[
    linecolor=red,
    bordercolor=red,
    backgroundcolor=white,
    textsize=tiny
]{todonotes}



\raggedbottom

% math commands
\usepackage{
  amscd,
  amsfonts,
  amsmath,
  amssymb,
  amsbsy,
  amsthm,
  bm, % boldmath
  dsfont,
  cancel, % cancellation
  latexsym,
  cases,
  mathtools,
  graphicx,
  xcolor,
  xargs,
  lineno
}

% % nepali characters for extra letters
% \usepackage{polyglossia}
% \newfontfamily\sanskrit[Script=Devanagari]{Lohit Devanagari}
% \newcommand{\Aa}{\textnormal{   \footnotesize \sanskrit अ }}
% \newcommand{\Ee}{\textnormal{   \footnotesize \sanskrit इ }}
% \newcommand{\Uu}{\textnormal{   \footnotesize \sanskrit ऊ }}
% \newcommand{\Ree}{\textnormal{  \footnotesize \sanskrit ऋ }}
% \newcommand{\Ey}{\textnormal{   \footnotesize \sanskrit ए }}
% \newcommand{\Ba}{\textnormal{   \footnotesize \sanskrit ब}}
% \newcommand{\Bha}{\textnormal{  \footnotesize \sanskrit भ}}
% \newcommand{\Ca}{\textnormal{   \footnotesize \sanskrit च}}
% \newcommand{\Cha}{\textnormal{  \footnotesize \sanskrit छ}}
% \newcommand{\Da}{\textnormal{   \footnotesize \sanskrit ड}}
% \newcommand{\Dda}{\textnormal{  \footnotesize \sanskrit द}}
% \newcommand{\Dha}{\textnormal{  \footnotesize \sanskrit ध}}
% \newcommand{\Ddha}{\textnormal{ \footnotesize \sanskrit ढ}}
% \newcommand{\Dna}{\textnormal{  \footnotesize \sanskrit ण}}
% \newcommand{\Ga}{\textnormal{   \footnotesize \sanskrit ग}}
% \newcommand{\Gha}{\textnormal{  \footnotesize \sanskrit घ}}
% \newcommand{\Ha}{\textnormal{   \footnotesize \sanskrit ह}}
% \newcommand{\Ja}{\textnormal{   \footnotesize \sanskrit ज}}
% \newcommand{\Jha}{\textnormal{  \footnotesize \sanskrit झ}}
% \newcommand{\Ka}{\textnormal{   \footnotesize \sanskrit क}}
% \newcommand{\Kha}{\textnormal{  \footnotesize \sanskrit ख}}
% \newcommand{\Ksha}{\textnormal{ \footnotesize \sanskrit ष}}
% \newcommand{\La}{\textnormal{   \footnotesize \sanskrit ल}}
% \newcommand{\Ma}{\textnormal{   \footnotesize \sanskrit म}}
% \newcommand{\Yna}{\textnormal{  \footnotesize \sanskrit ञ}}
% \newcommand{\Na}{\textnormal{   \footnotesize \sanskrit न}}
% \newcommand{\Nga}{\textnormal{  \footnotesize \sanskrit ङ}}
% \newcommand{\Pa}{\textnormal{   \footnotesize \sanskrit प}}
% \newcommand{\Pha}{\textnormal{  \footnotesize \sanskrit फ}}
% \newcommand{\Ra}{\textnormal{   \footnotesize \sanskrit र}}
% \newcommand{\Sa}{\textnormal{   \footnotesize \sanskrit स}}
% \newcommand{\Sha}{\textnormal{  \footnotesize \sanskrit श}}
% \newcommand{\Ta}{\textnormal{   \footnotesize \sanskrit ट}}
% \newcommand{\Tta}{\textnormal{  \footnotesize \sanskrit त}}
% \newcommand{\Tha}{\textnormal{  \footnotesize \sanskrit थ}}
% \newcommand{\Ttha}{\textnormal{ \footnotesize \sanskrit ठ}}
% \newcommand{\Va}{\textnormal{   \footnotesize \sanskrit व}}
% \newcommand{\Ya}{\textnormal{   \footnotesize \sanskrit य}}


% algorithms package
\usepackage{algorithm2e}
\SetKwInput{kwParam}{Parameters}

\newcommand{\figinc}[1]{\includegraphics[width=0.8\textwidth,keepaspectratio]{#1}}

\newcommand{\secend}{\vspace{3mm}}
\DeclareMathOperator{\erf}{erf}
% eqn environment
\newcommand{\beq}{\begin{equation}}
\newcommand{\eeq}{\end{equation}}
\newcommand{\link}[1]{\href{#1}{#1}}

% remaps for better greek letters
\newcommand{\epsi}{\varepsilon}
\newcommand{\phii}{\varphi}

% big parentheses
\newcommand*\Bigpar[1]{\left( #1 \right )}
\newcommand*\Bigbr[1]{\left[ #1 \right ]}
\newcommand*\Bigcr[1]{\left\{ #1 \right \}}
% set builder notation
\newcommand*\SetB[1]{\left\{ #1 \right\}}
% set
\newcommand*\Sett[1]{\mathcal{#1}}
% shorthand for data (Murphy PML style)
\newcommand{\Data}{\mathcal{D}}

% underbrace and overbrace
\newcommand{\Ubr}[2]{\underbrace{#1}_{\text{#2}}}
\newcommand{\Obr}[2]{ \overbrace{#1}^{\text{#2}}}

% command shorthand
\newcommand{\eg}{e.g., \xspace}
\newcommand{\ie}{i.e.,\xspace}
\newcommand{\etc}{etc.\@\xspace}
\newcommand{\iid}{\emph{i.i.d.}\ }
\newcommand{\etal}{et.\ al.\ }
\newcommand{\wprob}{\text{w.p.}}
\newcommand{\D}{\displaystyle}
\newcommand{\ba}{\begin{array}}
\newcommand{\ea}{\end{array}}
\newcommand{\be}{\begin{enumerate}}
\newcommand{\ee}{\end{enumerate}}
\newcommand{\bi}{\begin{itemize}}
\newcommand{\ei}{\end{itemize}}
\newcommand{\bs}{\begin{align}\begin{split}\nonumber}
\newcommand{\bsnumber}{\begin{align}\begin{split}}
\newcommand{\es}{\end{split}\end{align}}
\newcommand{\fns}{\singlespace\footnotesize}

\def\ST{\text{\; s.t. \;}}
\def\WP{\text{\; w.p. \;}}
\def\IF{\text{ if \;}}
\def\AS{\text{ as }}
\def\OR{\text{ or }}
\def\FOR{\text{ for }}
\def\AND{\text{ and }}
\def\THEN{\text{ then }}
\def\WITH{\text{ with }}
\newcommand{\WHERE}{\text{ where }}
\newcommand{\suchthat}{\text{ s.t. }}
\newcommand{\wrt}{\text{ w.r.t. }}
\providecommand{\subjectto}{\mathop{\rm subject\;to}}
\providecommand{\sto}{\mathop{\rm subject\;to}}
% ##     ##    ###    ######## ##     ##
% ###   ###   ## ##      ##    ##     ##
% #### ####  ##   ##     ##    ##     ##
% ## ### ## ##     ##    ##    #########
% ##     ## #########    ##    ##     ##
% ##     ## ##     ##    ##    ##     ##
% ##     ## ##     ##    ##    ##     ##


% derivative

\newcommandx{\deriv}[2][1=x,2=f]{\nabla \, #2 \Bigpar{ #1 } }
\newcommandx{\ortho}[1][1=L]{#1^{\bot}}


\newcommandx*\seqq[3][1=1,2=x, 3=n]{#2_{#1},\ldots,#2_{#3}}
\newcommandx*\coord[3][1=1,2=x, 3=n]{(#2_{#1},\ldots,#2_{#3})}

% Sums and products
\newcommand{\sumin}{\ensuremath{\sum_{i=1}^n}}
\newcommand{\sumiN}{\ensuremath{\sum_{i=1}^n}}
\newcommand{\sumim}{\ensuremath{\sum_{i=1}^m}}
\newcommand{\sumjk}{\ensuremath{\sum_{j=1}^k}}
\newcommand{\sumjj}{\ensuremath{\sum_{j=1}^J}}
\newcommand{\sumjn}{\ensuremath{\sum_{j=1}^n}}
\newcommand{\sumjm}{\ensuremath{\sum_{j=1}^m}}
\newcommand{\isum}[1][n]{\ensuremath{\sum_{#1}^\infty}}
\newcommand{\dsum}[4][i=1]{\ensuremath{\sum_{#1}^{#2}\sum_{#3}^{#4}}}
\newcommand{\Prod}[2][i=1]{\ensuremath{\prod_{#1}^{#2}}}
\newcommand{\prodin}{\ensuremath{\prod_{i=1}^N}}
\newcommand{\prodjn}{\ensuremath{\prod_{j=1}^n}}

% partial derivative
\newcommand{\dydx}[2]{\frac{\partial #1}{\partial #2}}
% vertical equal prefix
\newcommand{\verteq}{\rotatebox{90}{$\,=$}}
% vertical equal to
\newcommand{\equalto}[2]{\underset{\scriptstyle\overset{\mkern4mu\verteq}{#2}}{#1}}
% nullspace
\newcommand{\nulls}{\mathrm{null}}
% range
\newcommand{\range}{\mathrm{range}}

\newcommand*\maxx[1]{\text{max}\SetB{#1}}
\newcommand*\minn[1]{\text{min}\SetB{#1}}

% maximise
\newcommand{\maximise}{\operatornamewithlimits{max}}
\providecommand{\maximize}{\mathop{\rm maximize}}
% minimise
\newcommand{\minimise}{\operatornamewithlimits{min}}
\providecommand{\minimize}{\mathop{\rm minimize}}
% maximiser
\newcommand{\argmax}{\operatornamewithlimits{arg\,max}}
% general operator
\newcommand\oper[2]{\operatornamewithlimits{#1}_{#2}}


% indicator function
\newcommand*\Indic[1]{\mathds{1}_{#1}}

% ellipsis
\newcommand{\tto}{,\ldots,}

% blackboard F
\def\Function{\mathbb{F}}

% Lagrangian
\def\Lagr{\mathcal{L}}

% n-dimensional Real
\def\Realn{\mathbb{R}^n}

% k-dimensional Real
\def\Rk{\mathbb{R}^k}

% real with argument
\newcommand{\Realm}[1]{\mathbb{R}^{#1}}

% P_n
\def\Probn{\mathbb{P}_n}

\def\rel{\,{\buildrel R \over \sim}\,}
% generic m x n matrix
\newcommand{\gmatrix}[1]{\begin{bmatrix} {#1}_{11} & \cdots &{#1}_{1n} \\ \vdots & \ddots & \vdots \\ {#1}_{m1} & \cdots &{#1}_{mn} \end{bmatrix}}
% generic n x 1 vector
\newcommand{\gvec}[1]{\begin{bmatrix} {#1}_{1} \\ \vdots \\ {#1}_{n} \end{bmatrix}}

% matrix inverse
\newcommand{\inv}[1]{ \left({#1} \right)^{-1}}

\newcommand{\invg}[1]{ #1^{-1}}

% matrix transprose
\newcommand{\trap}[1]{ #1^{\top}}


\newcommand{\tra}{^{\top}}

% 2 X 2 matrix
\newcommand{\mattwo}[4]
{\left(\begin{array}{cc}
                        #1  & #2   \\
                        #3 &  #4
                          \end{array}\right) }

\newcommand{\matthree}[9]
{\left(\begin{array}{ccc}
                        #1  & #2 & #3  \\
                        #4 &  #5 & #6 \\
                        #7 &  #8 & #9
                          \end{array}\right) }

\newcommand{\dettwo}[4]
{\left|\begin{array}{cc}
                        #1  & #2   \\
                        #3 &  #4
                          \end{array}\right| }

\newcommand{\detthree}[9]
{\left|\begin{array}{ccc}
                        #1  & #2 & #3  \\
                        #4 &  #5 & #6 \\
                        #7 &  #8 & #9
                          \end{array}\right| }



% inner product
\newcommand{\iprod}[1]{\left\langle {#1} \right\rangle}
% vector norm
\newcommand{\norm}[1]{\left\Vert{#1} \right\Vert}

% Trace
\newcommand{\trace}[1]{\text{tr} \left({#1} \right)}

% absolute value
\newcommand{\abs}[1]{\left\vert {#1} \right\vert}
% linalg misc
\renewcommand{\det}{\mathrm{det}}
\newcommand{\rank}{\mathrm{rank}}
\newcommand{\trc}{\mathrm{trace}}
\newcommand{\spn}{\mathrm{span}}
\newcommand{\row}{\mathrm{Row}}
\newcommand{\col}{\mathrm{Col}}
\renewcommand{\dim}{\mathrm{dim}}
% weakly prefer
\newcommand{\prefeq}{\succeq}
% strictly prefer
\newcommand{\pref}{\succ}
% sequence
\newcommand{\seq}[1]{\{{#1}_n \}_{n=1}^\infty }
% single arrow
\renewcommand{\to}{{\rightarrow}}
% double arrow
\newcommand{\corres}{\overrightarrow{\rightarrow}}
% evaluate at
\newcommand*\Eval[2]{\left.#1\right\rvert_{#2}}
% evaluate definite integral
\newcommand*\IntEval[3]{\left.#1\right\rvert_{#2}^{#3}}


%Blackboard Letters
\newcommand{\R}{\ensuremath{\mathbb{R}}}
\newcommand{\Z}{\ensuremath{\mathbb{Z}}}
\newcommand{\Q}{\mathbb{Q}}
\newcommand{\N}{\mathbb{N}}
\newcommand{\W}{\mathbb{W}}
\newcommand{\Qoft}{\mathbb{Q}(t)}  %use in linux

\newcommand\frakfamily{\usefont{U}{yfrak}{m}{n}}
\DeclareTextFontCommand{\textfrak}{\frakfamily}

% Fractions
\newcommand{\ooN}{\frac{1}{n}}  %oneforth
\newcommand{\fof}{\frac{1}{4}}  %oneforth
\newcommand{\foh}{\frac{1}{2}}  %onehalf
\newcommand{\fot}{\frac{1}{3}}  %onethird
\newcommand{\fop}{\frac{1}{\pi}}    %1/pi
\newcommand{\ftp}{\frac{2}{\pi}}    %2/pi
\newcommand{\fotp}{\frac{1}{2 \pi}} %1/2pi
\newcommand{\fotpi}{\frac{1}{2 \pi i}}
\newcommand{\cm}{c_{\text{{\rm mean}}}}
\newcommand{\cv}{c_{\text{{\rm variance}}}}

% minimiser
\newcommand{\argmin}{\operatornamewithlimits{arg\,min}}
% convergence in probability sideways
\def\inprobLOW{\rightarrow_p}
% convergence in probability
\def\inprobHIGH{\,{\buildrel p \over \rightarrow}\,}
% convergence in probability 2
\def\inprob{\,{\inprobHIGH}\,}
% convergence in distribution
\def\indist{\,{\buildrel d \over \rightarrow}\,}

\def\asurely{\,{\buildrel a.s. \over \rightarrow}\,}


% definition bench
\newcommand{\defeq}{\vcentcolon=}
\newcommand{\eqdef}{=\vcentcolon}

% blackboard F
\def\Function{\mathbb{F}}
% Natural
\def\Nat{\mathbb{N}}
% Integers
\def\Int{\mathbb{Z}}
% Reals
\def\Real{\mathbb{R}}
% Rationals
\def\Rat{\mathbb{Q}}
% Complex
\def\Cplx{\mathbb{C}}

% n-dimensional Real
\def\Realn{\mathbb{R}^n}
% expectation_n
\def\Expn{\mathbb{E}_n}

% P_n
\def\Probn{\mathbb{P}_n}
\def\rel{\,{\buildrel R \over \sim}\,}

\renewcommand{\det}{\mathrm{det}}
\renewcommand{\dim}{\mathrm{dim}}
% single arrow
\renewcommand{\to}{{\rightarrow}}

% generic vector and matrix
\newcommand{\mc}[1]{\mathcal{#1}}
\def\mbf#1{\mathbf{#1}}
\def\mrm#1{\mathrm{#1}}
\def\mbi#1{\boldsymbol{#1}} % Bold and italic (math bold italic)
\def\ve#1{\mbi{#1}} % Vector notation
\def\vee#1{\mathbf{#1}} % Vector notation
\def\vea#1{\overrightarrow{#1}} % Vector notation

\newcommand*{\Vect}[1]{\mbi{#1}}
\newcommand*{\Mat}[1]{\mathbf{#1}}

\newcommand{\eucN}[1]{\norm{#1}} % l1 norm

\newcommand{\lzero}[1]{\norm{#1}_0} % l0 norm
\newcommand{\lone}[1]{\norm{#1}_1} % l1 norm
\newcommand{\ltwo}[1]{\norm{#1}_2} % l2 norm
\newcommand{\pnorm}[1]{\norm{#1}_p} % p-norm
\newcommand{\linf}[1]{\norm{#1}_\infty} % l-infinity norm
\newcommand{\lfro}[1]{\left\|{#1}\right\|_{\rm Fr}} % Frobenius norm
\newcommand{\matrixnorm}[1]{\left|\!\left|\!\left|{#1}
  \right|\!\right|\!\right|} % Matrix norm with three bars
\newcommand{\matrixnorms}[1]{|\!|\!|{#1}|\!|\!|} % Small matrix norm
\newcommand{\opnorm}[1]{\matrixnorm{#1}_{\rm op}}
\newcommand{\opnorms}[1]{\matrixnorms{#1}_{\rm op}}
\newcommand{\normbigg}[1]{\bigg\|{#1}\bigg\|} % A norm with 1 argument and bigg
                                              % brackets.
\newcommand{\lonebigg}[1]{\normbigg{#1}_1} % l1 norm
\newcommand{\ltwobigg}[1]{\normbigg{#1}_2} % l2 norm
\newcommand{\linfbigg}[1]{\normbigg{#1}_\infty} % l-infinity norm
\newcommand{\norms}[1]{\|{#1}\|} % A norm with 1 argument and normal (small)
                                 % brackets.
\newcommand{\lones}[1]{\norms{#1}_1} % l1 norm with small brackets
\newcommand{\ltwos}[1]{\norms{#1}_2} % l2 norm with small brackets
\newcommand{\linfs}[1]{\norms{#1}_\infty} % l-infinity norm with small brackets

\newcommand{\hinge}[1]{\left[{#1}\right]_+}

%%%%%%%%%%%%%%%%%%%%%%%%%%%%%%%%%%%%%%%%%%%%%%%%%%%%%%%%%%%%%%%%%%%%%
% tildes / hats / bars etc
%%%%%%%%%%%%%%%%%%%%%%%%%%%%%%%%%%%%%%%%%%%%%%%%%%%%%%%%%%%%%%%%%%%%%

\newcommand{\what}[1]{\widehat{#1}} % Wide hat
\newcommand{\wt}[1]{\widetilde{#1}} % Wide tilde
\newcommand{\wb}[1]{\overline{#1}} % Wide bar
% widehat, widetilde
\newcommand*\wtil[1]{\widetilde{#1}}
% over and underbars
\newcommand*\Ol[1]{\overline{#1}}
\newcommand*\Ul[1]{\underline{#1}}
% star
\newcommand*\Str[1]{#1^{*}}

\newcommand{\half}{\frac{1}{2}}

\newcommand{\<}{\left\langle} % Angle brackets
\renewcommand{\>}{\right\rangle} % End angle brackets

\renewcommand{\iff}{\Leftrightarrow}
\renewcommand{\choose}[2]{\binom{#1}{#2}}  % Choose
\newcommand{\chooses}[2]{{}_{#1}C_{#2}}  % Small choose

%%%% Probability symbols and associated distances %%%%

\newcommand{\E}{\mathbb{E}} % Expectation symbol
\renewcommand{\P}{\mathbb{P}} % Probability symbol
\newcommand{\var}{{\rm Var}} % Variance
\newcommand{\cov}{\mathop{\rm Cov}} % Covariance
\newcommand{\simiid}{\stackrel{\rm iid}{\sim}}
\newcommand{\openleft}[2]{\left({#1},{#2}\right]} % Interval open on left
\newcommand{\openright}[2]{\left[{#1},{#2}\right)} % Interval open on right

\newcommand{\indic}[1]{\mbf{1}\left\{#1\right\}} % Indicator function

% Distances between probability measures
\newcommand{\tvnorm}[1]{\norm{#1}_{\rm TV}} % Total variation
\newcommand{\tvnorms}[1]{\norms{#1}_{\rm TV}}
\newcommand{\dkl}[2]{D_{\rm kl}\left({#1} |\!| {#2}\right)} % KL divergence
\newcommand{\dkls}[2]{D_{\rm kl}({#1} |\!| {#2})}  % Small KL-divergence
\newcommand{\dchi}[2]{D_{\chi^2}\left({#1} |\!| {#2}\right)}  % chi^2-divergence
\newcommand{\fdiv}[2]{D_f\left({#1} |\!| {#2}\right)} % f divergence
\newcommand{\kl}[2]{D_{\rm kl}\left({#1} |\!| {#2} \right)}
\newcommand{\dhel}{d_{\rm hel}}  % Hellinger distance
\newcommand{\helaff}{A_{\rm hel}}  % Hellinger affinity



% Simple floor/ceiling stuff
\newcommand{\floor}[1]{\left\lfloor{#1} \right\rfloor}
\newcommand{\ceil}[1]{\left\lceil{#1} \right\rceil}

\providecommand{\argmax}{\mathop{\rm argmax}} % Defining math symbols
\providecommand{\argmin}{\mathop{\rm argmin}}
\providecommand{\soup}{\mathop{\rm sup}}

%  linalg stuff
\providecommand{\dom}{\mathop{\rm dom}}
\providecommand{\diag}{\mathop{\rm diag}}
\providecommand{\tr}{\mathop{\rm tr}}
\providecommand{\abs}{\mathop{\rm abs}}
\providecommand{\card}{\mathop{\rm card}}
\providecommand{\sign}{\mathop{\rm sign}}
\providecommand{\cl}{\mathop{\rm cl}}
\providecommand{\interior}{\mathop{\rm int}}
\providecommand{\conv}{\mathop{\rm Conv}}
\providecommand{\relint}{\mathop{\rm relint}}
\providecommand{\vol}{\mathop{\rm Vol}}
\providecommand{\supp}{\mathop{\rm supp}}


% Highlight part of eqn with colour
\newcommand*\hlred[1]{\textcolor{red}{#1}}
\newcommand*\hlblu[1]{\textcolor{blue}{#1}}


%  ######  ########    ###    ########  ######
% ##    ##    ##      ## ##      ##    ##    ##
% ##          ##     ##   ##     ##    ##
%  ######     ##    ##     ##    ##     ######
%       ##    ##    #########    ##          ##
% ##    ##    ##    ##     ##    ##    ##    ##
%  ######     ##    ##     ##    ##     ######


% data - curly D - Murphy notation
\def\Dat{\mathcal{D}}

% convergence in probability sideways
\def\inprobLOW{\rightarrow_p}
% convergence in probability
\def\inprobHIGH{\,{\buildrel p \over \rightarrow}\,}
% convergence in probability 2
\def\inprob{\,{\inprobHIGH}\,}
% convergence in distribution
\def\indist{\,{\buildrel d \over \rightarrow}\,}
% probability limit
\def\plim{\text{plim} \;}
% lim n
\def\limn{\text{plim}_{n \to \infty} \;}

% equality in distribution
\def\eqdist{\,{\buildrel d \over = }\,}

% independence (bench)
\newcommand\indep{\protect\mathpalette{\protect\independenT}{\perp}}
\def\independenT#1#2{\mathrel{\rlap{$#1#2$}\mkern5mu{#1#2}}}

% Likelihood
\newcommand{\Likl}{\mathcal{L}}
\newcommand{\Score}{\mathsf{S}}
\newcommand{\Hessian}{\mathsf{H}}

% bigO
\newcommand{\lilO}[1]{\mathsf{o}\Bigpar{#1}}
\newcommand{\bigO}[1]{\mathsf{O}\Bigpar{#1}}

% expectation
\newcommand{\Exp}[1]{\mathbb{E}\left[#1\right]}
% expectation at time
\newcommand{\Expt}[2]{\mathbb{E}_{#1}\left[#2\right]}
% expected utility for agent
\newcommand{\Expu}[1]{\mathbb{E}U^{#1}}
\newcommand{\Expua}[2]{\mathbb{E}U^{#1}\left[#2\right]}
\newcommand{\Expuat}[3]{\mathbb{E}U^{#1}_{#2}\left[#3\right]}
% variance
\newcommand{\Var}[1]{\mathbb{V}\left[#1\right]}
% covariance
\newcommand{\Covar}[1]{\text{Cov}\left[#1\right]}
% Probability
\newcommand{\Prob}[1]{\mathbf{Pr}\left(#1\right)}
% support
\newcommand{\Supp}[1]{\text{Supp}\left[#1\right]}
% do y | X = x
\newcommand{\doyx}{\Prob{Y \, |\,\mathsf{do} (X = x)}}
% do
\newcommand{\doo}[1]{\Prob{Y |\,\mathsf{do} (#1) }}

\newcommand{\larr}{\leftarrow}
\newcommand{\rarr}{\rightarrow}

% generic estimators
\newcommand{\betahat}{\hat{\beta}}
\newcommand{\thetahat}{\hat{\theta}}

% CDF
\newcommand{\F}{\mathbb{F}}
\newcommand{\FF}{\mathbb{F}}
\newcommand{\ff}{\mathsf{f}}
\newcommand{\Cdf}[1]{\mathbb{F}\left(#1\right)}
\newcommand{\Cdff}[2]{\mathbb{F}_{#1}\left(#2\right)}
% PDF
\newcommand{\Pdf}[1]{\mathsf{f}\left(#1\right)}
\newcommand{\Pdff}[2]{\mathsf{f}_{#1}\left(#2\right)}
% derivative
\newcommand{\dd}{\mathsf{d}}



% Convergence of random variables
\newcommand{\cd}{\stackrel{d}{\rightarrow}}
\newcommand{\cas}{\stackrel{a.s.}{\rightarrow}}
\newcommand{\cp}{\stackrel{p}{\rightarrow}}

\newcommand{\rootn}{\sqrt{n}}

% Probability distributions
\newcommand*\normal[1]{\mathcal{N} \left( #1 \right )}
\newcommand*\Normal[1]{\mathcal{N} \left( #1 \right )}
\newcommand*\Normalp[2]{\mathcal{N}_{#1} \left( #2 \right )}
% \newcommand{\unif}{\mathsf{U}}  % Uniform distribution
\newcommand*\Unif[1]{\mathsf{U} \left[ #1 \right ]}
\newcommand*\unif[1]{\mathsf{U} \left[ #1 \right ]}
% Bernoulli
\newcommand*\Bern[1]{\mathsf{Bernoulli} \left( #1 \right )}
\newcommand*\Binom[1]{\mathsf{Bin} \left( #1 \right )}
% Poisson
\newcommand*\Pois[1]{\mathsf{Poi} \left( #1 \right )}
% beta
\newcommand*\BetaD[1]{\mathsf{Beta} \left( #1 \right )}
% dirichlet
\newcommand*\Diri[1]{\mathsf{Dir} \left( #1 \right )}
% gamma
\newcommand*\Gdist[1]{\mathsf{Gamma} \left( #1 \right )}
% inv chi squared
\newcommand*\InvChi[1]{\mathsf{Inv}-\chi^2 \left( #1 \right )}

\def\bias{\textsf{bias}\xspace}
\def\se{\textsf{se}\xspace}

\def\pdf{\textsc{pdf}\xspace}
\def\cdf{\textsc{cdf}\xspace}
\def\ise{\textsc{ise}\xspace}
\def\pgf{\textsc{pgf}\xspace}
\def\mgf{\textsc{mgf}\xspace}
\def\mse{\textsc{mse}\xspace}
\def\mspe{\textsc{mspe}\xspace}
\def\mle{\textsc{mle}\xspace}
\def\mom{\textsc{mom}\xspace}
\def\are{\textsc{are}\xspace}
\def\rss{\textsc{rss}\xspace}
\def\ess{\textsc{ess}\xspace}
\def\tss{\textsc{tss}\xspace}


% Naming shortcuts.
\def\ahat{\ensuremath{\widehat{\alpha}}}
\def\atil{\ensuremath{\tilde{\alpha}}}
\def\bhat{\ensuremath{\widehat{\beta}}}
\def\btil{\ensuremath{\tilde{\beta}}}
\def\dhat{\ensuremath{\widehat{\delta}}}
\def\ehat{\ensuremath{\hat{\epsilon}}}
\def\ghat{\ensuremath{\widehat{\gamma}}}
\def\khat{\ensuremath{\widehat{\kappa}}}
\def\lhat{\ensuremath{\widehat{\lambda}}}
\def\ltil{\ensuremath{\tilde{\lambda}}}
\def\mhat{\ensuremath{\widehat{\mu}}}
\def\nhat{\ensuremath{\widehat{\nu}}}
\def\mtil{\ensuremath{\tilde{\mu}}}
\def\psihat{\ensuremath{\widehat{\psi}}}
\def\shat{\ensuremath{\widehat{\sigma}}}
\def\stil{\ensuremath{\tilde{\sigma}}}
\def\that{\ensuremath{\widehat{\theta}}}
\def\ttil{\ensuremath{\widetilde{\theta}}}
\def\rhohat{\widehat{\rho}}
\def\xihat{\widehat{\xi}}

\def\sehat{\ensuremath{\widehat{\se}}}
\def\fhat{\ensuremath{\widehat{f}}}
\def\Fhat{\ensuremath{\widehat{F}}}
\def\fnhat{\ensuremath{\widehat{f}_n}}
\def\Fnhat{\ensuremath{\widehat{F}_n}}
\def\Jhat{\ensuremath{\widehat{J}}}
\def\phat{\ensuremath{\widehat{p}}}
\def\ptil{\ensuremath{\tilde{p}}}
\def\rhat{\widehat{r}}
\def\Rbar{\bar{R}}
\def\Rhat{\widehat{R}}
\def\Qbar{\bar{Q}}
\def\Qhat{\widehat{Q}}
\def\Xhat{\widehat{X}}
\def\xbar{\bar{x}}
\def\Xbar{\bar{X}}
\def\Xsqbar{\overline{X^2}}
\def\xnbar{\overline{x}_n}
\def\Xnbar{\overline{X}_n}
\def\Yhat{\widehat{Y}}
\def\ybar{\overline{y}}
\def\Ybar{\overline{Y}}
\def\Ynbar{\overline{Y}_n}

% Random variables.
\def\rv{\textsc{rv}\xspace}
\def\iid{\ensuremath{\textsc{iid}}\xspace}
\def\dist{\ensuremath{\sim}\xspace}
\def\disteq{\ensuremath{\stackrel{D}{=}}\xspace}
\def\distiid{\ensuremath{\stackrel{iid}{\sim}}\xspace}
\def\ind{\ensuremath{\perp\!\!\!\perp}\xspace}
\def\nind{\ensuremath{\perp\!\!\!\!\big\vert\!\!\!\!\perp}\xspace}
\def\Xon{\ensuremath{X_1,\dots,X_n}\xspace}
\def\xon{\ensuremath{x_1,\dots,x_n}\xspace}
\def\giv{\ensuremath{\,|\,}}
\def\Giv{\ensuremath{\,\big|\,}}
\def\GIV{\ensuremath{\,\Big|\,}}
\newcommand{\indicator}[1]{\mathds{1}_{\left\{#1\right\}}}

% Probability, expectation, and variance.
\def\prob{\mathbb{P}}
\renewcommand{\Pr}[2][]{\ensuremath{\prob_{#1}\left[#2\right]}\xspace}
\newcommand{\V}[2][]{\ensuremath{\mathbb{V}_{#1}\left[#2\right]}}
\newcommand{\corr}[2][]{\ensuremath{\rho_{#1}\left[#2\right]}}
\def\sd{\ensuremath{\textsf{sd}}\xspace}
\def\samplemean{\ensuremath{\bar{X}_n}\xspace}
\def\samplevar{\ensuremath{S^2}\xspace}
\def\za{\ensuremath{z_{\alpha}}}
\def\zat{\ensuremath{z_{\alpha/2}}}

% Inference
\def\Ll{\ensuremath{\mathcal{L}}\xspace}
\def\Lln{\ensuremath{\Ll_n}\xspace}
\def\ll{\ensuremath{\ell}}
\def\lln{\ensuremath{\ll_n}}

% Hypothesis testing
\newcommand{\hyp}[2]{
\ensuremath{H_0:#1 \ifhmode\quad\text{versus}\quad\fi\text{ vs. } H_1:#2}}

% Convergence.
\def\conv{\rightarrow}
\def\convinf{\rightarrow_{n\to\infty}}
\def\pconv{\stackrel{\text{\tiny{P}}}{\rightarrow}}
\def\npconv{\stackrel{\text{\tiny{P}}}{\nrightarrow}}
\def\dconv{\stackrel{\text{\tiny{D}}}{\rightarrow}}
\def\ndconv{\stackrel{\text{\tiny{D}}}{\nrightarrow}}
\def\qmconv{\stackrel{\text{\tiny{qm}}}{\rightarrow}}
\def\nqmconv{\stackrel{\text{\tiny{qm}}}{\nrightarrow}}
\def\asconv{\stackrel{\text{\tiny{as}}}{\rightarrow}}
\def\nasconv{\stackrel{\text{\tiny{as}}}{\nrightarrow}}

%
% Distributions
%
\newcommandx{\uniff}[1][1={a,b}]{\textrm{Unif}\left({#1}\right)}
\newcommandx{\unifd}[1][1={a,\ldots,b}]{\textrm{Unif}\left\{{#1}\right\}}
\newcommandx{\dunif}[3][1=x,2=a,3=b]{\frac{I(#2<#1<#3)}{#3-#2}}
\newcommandx{\dunifd}[3][1=x,2=a,3=b]{\frac{I(#2\le#1\le#3)}{#3-#2+1}}
\newcommandx{\punif}[3][1=x,2=a,3=b]{
\begin{cases} 0 & #1 < #2 \\ \frac{#1-#2}{#3-#2} & #2 < #1 < #3 \\ 1 & #1 > #3\\\end{cases}}
\newcommandx{\punifd}[3][1=x,2=a,3=b]{
\begin{cases} 0 & #1 < #2\\ \frac{\lfloor#1\rfloor-#2+1}{#3-#2} & #2 \le #1 \le #3 \\ 1 & #1 > #3\\ \end{cases}}

% Bernoulli
\newcommandx\bern[1][1=p]{\textrm{Bern}\left({#1}\right)}
\newcommandx\dbern[2][1=x,2=p]{#2^{#1} \left(1-#2\right)^{1-#1}}
\newcommandx\pbern[2][1=x,2=p]{\left(1-#2\right)^{1-#1}}

% Binomial
\newcommandx\bin[1][1={n,p}]{\textrm{Bin}\left(#1\right)}
\newcommandx\dbin[3][1=x,2=n,3=p]{\binom{#2}{#1}#3^#1\left(1-#3\right)^{#2-#1}}

% Multinomial
\newcommandx\mult[1][1={n,p}]{\textrm{Mult}\left(#1\right)}
\newcommandx\dmult[3][1=x,2=n,3=p]{\frac{#2!}{#1_1!\ldots#1_k!}#3_1^{#1_1}\cdots#3_k^{#1_k}}

% Hypergeometric
\newcommandx\hyper[1][1={N,m,n}]{\textrm{Hyp}\left({#1}\right)}
\newcommandx\dhyper[4][1=x,2=N,3=m,4=n]{\frac{\binom{#3}{#1}\binom{#2-#3}{#4-#1}}{\binom{#2}{#4}}}

% Negative Binomial
\newcommandx\nbin[1][1={r,p}]{\textrm{NBin}\left({#1}\right)}
\newcommandx\dnbin[3][1=x,2=r,3=p]{\binom{#1+#2-1}{#2-1}#3^#2(1-#3)^#1}
\newcommandx\pnbin[3][1=x,2=r,3=p]{I_#3(#2,#1+1)}

% Geometric
\newcommandx\geo[1][1=p]{\textrm{Geo}\left(#1\right)}
\newcommandx\dgeo[2][1=x,2=p]{#2(1-#2)^{#1-1}}
\newcommandx\pgeo[2][1=x,2=p]{1-(1-#2)^#1}

% Poisson
\newcommandx\pois[1][1=\lambda]{\textrm{Po}\left({#1}\right)}
\newcommandx\dpois[2][1=x,2=\lambda]{\frac{#2^#1 e^{-#2}}{#1!}}
\newcommandx\ppois[2][1=x,2=\lambda]{e^{-#2}\sum_{i=0}^#1\frac{#2^i}{i!}}

% Normal
\newcommandx\normall[1][1={\mu,\sigma^2}]{\mathcal{N}\left({#1}\right)}
\newcommandx\dnormall[3][1=x,2=\mu,3=\sigma]%
  {\frac{1}{#3\sqrt{2\pi}}\exp \Bigpar{-\frac{\left(#1-#2\right)^2}{2 #3^2}}}
\newcommandx\pnormall[1][1=x]{\Phi\left({#1}\right)}
\newcommandx\qnormall[1]{\Phi^{-1}\left({#1}\right)}

% Multivariate Normal
\newcommandx\mvn[1][1={\mu,\Sigma}]{\mathrm{MVN}\left({#1}\right)}

% Exponential
\newcommandx\ex[1][1=\lambda]{\textrm{Exp}\left(#1\right)}
\newcommandx\dex[2][1=x,2=\lambda]{#2e^{-#1 #2}}
\newcommandx\pex[2][1=x,2=\lambda]{1-e^{-#1 #2}}

% Gamma
\newcommandx\gam[1][1={\alpha,\lambda}]{\textrm{Gamma}\left({#1}\right)}
\newcommandx\dgamma[3][1=x,2=\alpha,3=\lambda]%
  {\frac{#3^{#2}}{\Gamma\left( #2 \right)} #1^{#2-1}e^{-#3#1}}

% InverseGamma
\newcommandx\invgamma[1][1={\alpha,\beta}]{\textrm{InvGamma}\left({#1}\right)}
\newcommandx\dinvgamma[3][1=x,2=\alpha,3=\beta]%
{\frac{#3^{#2}}{\Gamma\left(#2\right)}#1^{-#2-1}e^{-#3/#1}}
\newcommandx\pinvgamma[3][1=x,2=\alpha,3=\beta]%
{\frac{\Gamma\left(#2,\frac{#3}{#1}\right)}{\Gamma\left(#2\right)}}

% Beta
\newcommandx\bet[1][1={\alpha,\beta}]{\textrm{Beta}\left(#1\right)}
\newcommandx\dbeta[3][1=x,2=\alpha,3=\beta]
{\frac{\Gamma\left(#2+#3\right)}{\Gamma\left(#2\right)\Gamma\left(#3\right)}#1^{#2-1}\left(1-#1\right)^{#3-1}}

% Dirichlet
\newcommandx\dir[1][1={\alpha}]{\textrm{Dir}\left(#1\right)}
\newcommandx\ddir[3][1=x,2=\alpha]{\frac{\Gamma\left(\sum_{i=1}^k #2_i\right)}{\prod_{i=1}^k\Gamma\left(#2_i\right)}\prod_{i=1}^k #1_i^{#2_i-1}}

% Weibull
\newcommandx\weibull[1][1={\alpha}]{\textrm{Dir}\left(#1\right)}
\newcommandx\dweibull[3][1=x,2=\lambda,3=k]{\frac{#3}{#2}
\left(\frac{#1}{#2}\right)^{#3-1} e^{-(#1/#2)^k}}

% Chi-squard
\newcommandx\chisq[1][1=k]{\chi_{#1}^2}
